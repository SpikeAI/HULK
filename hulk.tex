%!TEX root = hulk.tex
%!TeX TS-program = Lualatex
%!TeX encoding = UTF-8 Unicode
%!TeX spellcheck = en-US
%!BIB TS-program = bibtex
% -*- coding: UTF-8; -*-
% vim: set fenc=utf-8
\documentclass[a4paper, 11pt]{article} % For LaTeX2e

% if you need to pass options to natbib, use, e.g.:
% \PassOptionsToPackage{numbers, compress}{natbib}
% before loading nips_2017
%
% to avoid loading the natbib package, add option nonatbib:
% \usepackage[nonatbib]{nips_2017}

\usepackage[nonatbib]{nips_2018}
%\input{CHAMP_Metadata}
% to compile a camera-ready version, add the [final] option, e.g.:
%\usepackage[final]{nips_2017}

\usepackage[utf8]{inputenc} % allow utf-8 input
\usepackage[T1]{fontenc}    % use 8-bit T1 fonts
\usepackage{hyperref}       % hyperlinks


%%%%%%%%%%%%%%%%%%%%%%%%%%%%%%%%%%%%%%%%%%%%%%%%%%%%%%%%%%%%%%%%%%%%%
\def\Draft{1}%
\def\Draft{0}%
%%%%%%%%%%%%%%%%%%%%%%%%%%%%%%%%%%%%%%%%%%%%%%%%%%%%%%%%%%%%%%%%%%%%%
% --------------------------------------------------------------------------
%                    METADATA
% --------------------------------------------------------------------------
\newcommand{\AuthorA}{Boutin}%
\newcommand{\FirstNameA}{Victor}%
\newcommand{\AuthorB}{Franciosini}%
\newcommand{\FirstNameB}{Angelo}%
\newcommand{\AuthorC}{Perrinet}%
\newcommand{\FirstNameC}{Laurent U}%
\newcommand{\Institute}{Institut Neurosciences Timone, Marseille, France}%\\ Aix Marseille Univ, CNRS}%
\newcommand{\Organism}{Aix Marseille Universit\'e, CNRS}%
\newcommand{\Address}{27, Bd. Jean Moulin, 13385 Marseille Cedex 5, France}%
\newcommand{\Website}{http://invibe.net/LaurentPerrinet}%
\newcommand{\EmailA}{victor.boutin@univ-amu.fr}%
\newcommand{\EmailB}{angelo.franciosini@univ-amu.fr}%
\newcommand{\EmailC}{laurent.perrinet@univ-amu.fr}%victor.boutin@univ-amu.fr
\newcommand{\WebsiteC}{http://invibe.net/LaurentPerrinet}%
\newcommand{\Title}{%
A fast algorithm for unsupervised learning
%alternatives: - Homeostasis is necessary for the unsupervised learning of orientation-selective cells
%# Homeostatic Unsupervised Learning of Kernels
}%
\newcommand{\Abstract}{
The formation of structure in the brain, that is, of the connection
between cells within neural populations, is by large an unsupervised
learning process: The emergence of this architecture is
mostly self-organized. In the primary visual cortex of mammals, for
example, one may observe during development the emergence of cells
selective to localized, oriented features. This leads to the development
of a rough representation of contours of the retinal image in area V1. A
major difficulty in defining unsupervised learning algorithms is that
during this process, on the one hand the coding is performed knowing an
immature structure and on the other hand, the adaptation of this
structure is carried out knowing a code that is not yet optimal. We
propose here a fast algorithm compatible with a neuromimetic
architecture which solves this problem and allows for the fast emergence
of localized filters sensitive to orientation. The key to this algorithm
lies in a simple yet optimal mechanism of homeostasis that reconciles
the antagonistic processes that occur at the coding and learning time scales.
We tested this unsupervised algorithm with this homeostasis rule for a
range of existing unsupervised learning algorithms coupled with
different neural coding algorithms. In addition, we propose a
simplification of this optimal homeostasis rule by implementing a simple
heuristic on the probability of activation of neurones. 
Compared to the optimal homeostasis rule, we show that this heuristic allows to
implement an even faster unsupervised learning algorithm while keeping a large
part of its effectiveness. 
These results demonstrate the potential application 
of such a strategy to the fast classification of images, 
for example in hierarchical and dynamic architectures.
}
\newcommand{\Keywords}{Retina, Sparseness, Computer vision, Unsupervised learning, Neuroscience}%
\newcommand{\Acknowledgments}{%
This work was supported by XXX and the Doc2Amu project which received funding from a co-fund with the European Union's Horizon 2020 research and innovation programme and the region Provence Alpes Cote d'Azur. }
\newcommand{\Links}{%
\begin{itemize}
\item Correspondence and requests for materials should be addressed to LUP (email:\EmailC ). 

\item Code and supplementary material available at \url{\WebsiteC/Publications/BoutinFranciosiniPerrinet18hulk}.
\end{itemize}
} %

% --------------------------------------------------------------------------
\usepackage{filecontents}

\begin{filecontents}{hulk.bib}

@article{olshausen1996emergence,
  title={Emergence of simple-cell receptive field properties by learning a sparse code for natural images},
  author={Olshausen, Bruno A and Field, David J},
  journal={Nature},
  volume={381},
  number={6583},
  pages={607},
  year={1996},
  publisher={Nature Publishing Group}
}

@article{Olshausen97,
	Abstract = {The spatial receptive fields of simple cells in mammalian striate cortex have been reasonably well described physiologically and can be characterized as being localized, oriented, and bandpass, comparable to the basis functions of wavelet transforms. Previously, we have shown that these receptive field properties may be accounted for in terms of a strategy for producing a sparse distribution of output activity in response to natural images (Olshausen and Field, 1996a). Here, in addition to describing this work in a more expansive fashion, we examine the neurobiological implications of sparse coding. Of particular interest is the case when the code is overcomplete---i.e., when the number of code elements is greater than the effective dimensionality of the input space. Because the basis functions are non-orthogonal and not linearly independent of each other, sparsifying the code will recruit only those basis functions necessary for representing a given input, and so the input-output fun...},
	Author = {Olshausen, Bruno A. and Field, David J.},
	Citeulike-Article-Id = {9026860},
	Citeulike-Linkout-0 = {http://dx.doi.org/10.1016/S0042-6989(97)00169-7},
	Citeulike-Linkout-1 = {http://citeseerx.ist.psu.edu/viewdoc/summary?doi=10.1.1.27.712},
	Date-Added = {2017-01-02 14:34:49 +0000},
	Date-Modified = {2017-01-02 14:34:49 +0000},
	Doi = {10.1016/S0042-6989(97)00169-7},
	Issn = {0042-6989},
	Journal = {Vision Research},
	Keywords = {area-v1, assofield, bicv-sparse, sparse\_hebbian\_learning},
	Month = dec,
	Number = {23},
	Pages = {3311--3325},
	Priority = {0},
	Title = {Sparse coding with an overcomplete basis set: A strategy employed by {V1}?},
	Url = {http://dx.doi.org/10.1016/S0042-6989(97)00169-7},
	Volume = {37},
	Year = {1997},
	Bdsk-Url-1 = {http://dx.doi.org/10.1016/S0042-6989(97)00169-7}}


@article{hubel1968receptive,
  title={Receptive fields and functional architecture of monkey striate cortex},
  author={Hubel, David H and Wiesel, Torsten N},
  journal={The Journal of {P}hysiology},
  volume={195},
  number={1},
  pages={215--243},
  year={1968},
  publisher={Wiley Online Library}
}


@article{Serre07,
abstract = {Primates are remarkably good at recognizing objects. The level of performance of their visual system and its robustness to image degradations still surpasses the best computer vision systems despite decades of engineering effort. In particular, the high accuracy of primates in ultra rapid object categorization and rapid serial visual presentation tasks is remarkable. Given the number of processing stages involved and typical neural latencies, such rapid visual processing is likely to be mostly feedforward. Here we show that a specific implementation of a class of feedforward theories of object recognition (that extend the Hubel and Wiesel simple-to-complex cell hierarchy and account for many anatomical and physiological constraints) can predict the level and the pattern of performance achieved by humans on a rapid masked animal vs. non-animal categorization task.},
author = {Serre, Thomas and Oliva, Aude and Poggio, Tomaso},
doi = {10.1073/pnas.0700622104},
file = {:Users/lolo/quantic/science/bibtexing/Serre, Oliva, Poggio/Serre, Oliva, Poggio_2007_A feedforward architecture accounts for rapid categorization.pdf:pdf},
issn = {1091-6490},
journal = {Proceedings of the National Academy of Sciences},
keywords = {assofield,bicv-sparse,perrinet11sfn,perrinetbednar15},
mendeley-tags = {assofield,perrinet11sfn},
number = {15},
pages = {6424--6429},
pmid = {17404214},
publisher = {National Academy of Sciences},
title = {{A feedforward architecture accounts for rapid categorization}},
type = {Journal article},
url = {http://dx.doi.org/10.1073/pnas.0700622104 http://www.pnas.org/content/104/15/6424.abstract http://www.pnas.org/content/104/15/6424.full.pdf http://www.pnas.org/cgi/content/abstract/104/15/6424},
volume = {104},
year = {2007}
}
@article{Carandini12,
abstract = {There is increasing evidence that the brain relies on a set of canonical neural computations, repeating them across brain regions and modalities to apply similar operations to different problems. A promising candidate for such a computation is normalization, in which the responses of neurons are divided by a common factor that typically includes the summed activity of a pool of neurons. Normalization was developed to explain responses in the primary visual cortex and is now thought to operate throughout the visual system, and in many other sensory modalities and brain regions. Normalization may underlie operations such as the representation of odours, the modulatory effects of visual attention, the encoding of value and the integration of multisensory information. Its presence in such a diversity of neural systems in multiple species, from invertebrates to mammals, suggests that it serves as a canonical neural computation.},
author = {Carandini, Matteo and Heeger, David J Dj},
doi = {10.1038/nrn3136},
isbn = {1471-0048 (Electronic)\n1471-003X (Linking)},
issn = {1471-003X},
journal = {Nature Reviews Neuroscience},
keywords = {Adaptation,Afferent Pathways,Anim,Physiological,contrast_response,divisive_normalization,normalization},
mendeley-tags = {contrast_response,divisive_normalization,normalization},
month = {jan},
number = {November},
pages = {1--12},
pmid = {22108672},
publisher = {Nature Publishing Group},
title = {{Normalization as a canonical neural computation.}},
type = {Journal article},
url = {http://discovery.ucl.ac.uk/1332718/ http://www.ncbi.nlm.nih.gov/pubmed/22108672 http://www.pubmedcentral.nih.gov/articlerender.fcgi?artid=PMC3273486},
volume = {13},
year = {2012}
}

@article{PerrinetBednar15,
	Abstract = {Making a judgment about the semantic category of a visual scene, such as whether it contains an animal, is typically assumed to involve high-level associative brain areas. Previous explanations require progressively analyzing the scene hierarchically at increasing levels of abstraction, from edge extraction to mid-level object recognition and then object categorization. Here we show that the statistics of edge co-occurrences alone are sufficient to perform a rough yet robust (translation, scale, and rotation invariant) scene categorization. We first extracted the edges from images using a scale-space analysis coupled with a sparse coding algorithm. We then computed the ``association field'' for different categories (natural, man-made, or containing an animal) by computing the statistics of edge co-occurrences. These differed strongly, with animal images having more curved configurations. We show that this geometry alone is sufficient for categorization, and that the pattern of errors made by humans is consistent with this procedure. Because these statistics could be measured as early as the primary visual cortex, the results challenge widely held assumptions about the flow of computations in the visual system. The results also suggest new algorithms for image classification and signal processing that exploit correlations between low-level structure and the underlying semantic category.},
	Author = {Perrinet, Laurent U. and Bednar, James A.},
	Doi = {10.1038/srep11400},
	Issn = {2045-2322},
	Journal = {Scientific Reports},
	Keywords = {anr-trax,assofield,bicv-sparse,perrinetbednar15,sanz12jnp,vacher14},
	Language = {en},
	Month = {jun},
	Pages = {11400},
	Pmid = {26096913},
	Publisher = {Nature Publishing Group},
	Title = {{Edge co-occurrences can account for rapid categorization of natural versus animal images}},
	Url = {http://www.nature.com/doifinder/10.1038/srep11400 http://www.nature.com/srep/2015/150622/srep11400/full/srep11400.html http://www.nature.com/articles/srep11400},
	Volume = {5},
	Year = {2015},
	Bdsk-Url-1 = {http://www.nature.com/doifinder/10.1038/srep11400%20http://www.nature.com/srep/2015/150622/srep11400/full/srep11400.html%20http://www.nature.com/articles/srep11400},
	Bdsk-Url-2 = {http://dx.doi.org/10.1038/srep11400}}


@article{Perrinet10shl,
	Abstract = {Neurons in the input layer of primary visual cortex in primates develop edge-like receptive fields. One approach to understanding the emergence of this response is to state that neural activity has to efficiently represent sensory data with respect to the statistics of natural scenes. Furthermore, it is believed that such an efficient coding is achieved using a competition across neurons so as to generate a sparse representation, that is, where a relatively small number of neurons are simultaneously active. Indeed, different models of sparse coding coupled with Hebbian learning and homeostasis have been proposed that successfully match the observed emergent response. However, the specific role of homeostasis in learning such sparse representations is still largely unknown. By quantitatively assessing the efficiency of the neural representation during learning, we derive a cooperative homeostasis mechanism which optimally tunes the competition between neurons within the sparse coding algorithm. We apply this homeostasis while learning small patches taken from natural images and compare its efficiency with state-of-the-art algorithms. Results show that while different sparse coding algorithms give similar coding results, the homeostasis provides an optimal balance for the representation of natural images within the population of neurons. Competition in sparse coding is optimized when it is fair: By contributing to optimize statistical competition across neurons, homeostasis is crucial in providing a more efficient solution to the emergence of independent components.},
	Annote = {Posted Online March 17, 2010.},
	Author = {Perrinet, Laurent U.},
	Citeulike-Article-Id = {7158387},
	Citeulike-Linkout-0 = {http://invibe.net/LaurentPerrinet/Publications/Perrinet10shl},
	Citeulike-Linkout-1 = {http://dx.doi.org/10.1162/neco.2010.05-08-795},
	Citeulike-Linkout-2 = {http://www.mitpressjournals.org/doi/abs/10.1162/neco.2010.05-08-795},
	Citeulike-Linkout-3 = {http://view.ncbi.nlm.nih.gov/pubmed/20235818},
	Citeulike-Linkout-4 = {http://www.hubmed.org/display.cgi?uids=20235818},
	Date-Added = {2017-01-02 13:44:42 +0000},
	Date-Modified = {2017-01-02 13:44:42 +0000},
	Day = {17},
	Doi = {10.1162/neco.2010.05-08-795},
	Issn = {1530-888X},
	Journal = {Neural Computation},
	Keywords = {adaptive, assofield, bicv-sparse, cell, coding, competition-optimized, cooperative, fields, hebbian, homeostasis, images, khoei13jpp, learning, matching, matching-pursuit, natural, natural-scenes, neural, of, overcomplete\_dictionaries, perrinet10shl, perrinet11sfn, perrinet12pred, population, pursuit, receptive, sanz12jnp, simple, sparse, sparse\_coding, sparse\_hebbian\_learning, sparse\_spike\_coding, statistics, unsupervised, vacher14},
	Month = jul,
	Number = {7},
	Pages = {1812--36},
	Pmid = {20235818},
	Priority = {0},
	Publisher = {MIT Press},
	Title = {Role of homeostasis in learning sparse representations},
	Url = {http://invibe.net/LaurentPerrinet/Publications/Perrinet10shl},
	Volume = {22},
	Year = {2010},
	Bdsk-Url-1 = {http://invibe.net/LaurentPerrinet/Publications/Perrinet10shl},
	Bdsk-Url-2 = {http://dx.doi.org/10.1162/neco.2010.05-08-795}}

@incollection{Perrinet15bicv,
	Author = {Perrinet, Laurent U.},
	Booktitle = {Biologically Inspired Computer Vision},
	Chapter = {13},
	Citeulike-Article-Id = {13566753},
	Date-Added = {2017-01-02 13:44:42 +0000},
	Date-Modified = {2017-01-02 13:44:42 +0000},
	Editor = {Crist{\'{o}}bal, Gabriel and Perrinet, Laurent and Keil, Matthias S.},
	Isbn = {9783527680863},
	Keywords = {bicv-sparse},
	Month = {nov},
	Priority = {2},
	Publisher = {Wiley-VCH Verlag},
	Title = {Sparse Models for Computer Vision},
	Url = {http://onlinelibrary.wiley.com/doi/10.1002/9783527680863.ch14/summary},
	Year = {2015},
	Bdsk-Url-1 = {http://onlinelibrary.wiley.com/doi/10.1002/9783527680863.ch14/summary}}


\end{filecontents}


\usepackage[utf8]{luainputenc}
%============ common ===================
\usepackage[english]{babel}%
%\usepackage{csquotes}%
\usepackage[autostyle]{csquotes}
%% Sans-serif Arial-like fonts
\renewcommand{\rmdefault}{phv} 
\renewcommand{\sfdefault}{phv} 
     \usepackage{textcomp}
     \usepackage{libertine}%[sb]
     \usepackage[varqu,varl]{inconsolata}% sans serif typewriter
     \usepackage[libertine,bigdelims,vvarbb]{newtxmath} % bb from STIX
     \usepackage[cal=boondoxo]{mathalfa} % mathcal
%     \useosf % osf for text, not math
     \usepackage[supstfm=libertinesups,%
       supscaled=1.2,%
       raised=-.13em]{superiors}
%%%%%%%%%%%%%%%%%%%%%%%%%%%%%%
%\usepackage[unicode,linkcolor=blue,citecolor=blue,filecolor=black,urlcolor=blue,pdfborder={0 0 0}]{hyperref}%
%\hypersetup{%
%pdftitle={\Title},%
%pdfauthor={Corrresponding author: \AuthorC < \EmailC > \Address - \WebsiteC },%
%pdfkeywords={\Keywords},%
%pdfsubject={\Acknowledgments}%
%}%
\usepackage{url}            % simple URL typesetting
\usepackage{booktabs}       % professional-quality tables
\usepackage{amsfonts}       % blackboard math symbols
\usepackage{nicefrac}       % compact symbols for 1/2, etc.
\usepackage{microtype}      % microtypography

\usepackage[dvipsnames]{xcolor}

\usepackage{tikz}%
%\if 1\Draft
%\usepackage{setspace}
%\fi
% MATHS (AMS)
%\usepackage{amsmath}
%%\usepackage{amsfonts}
%\usepackage{amssymb}
%\usepackage{amsthm}
%\usepackage{amsfonts, amssymb, amscd}
\newcommand{\norm}[1]{|\!| #1 |\!|}
\newcommand{\dotp}[2]{\langle #1,\,#2\rangle}
\newcommand{\eqdef}{\ensuremath{\stackrel{\mbox{\upshape\tiny def.}}{=}}}
\newcommand{\eqset}{\ensuremath{\stackrel{\mbox{\upshape\tiny set}}{=}}}
\newcommand{\eq}[1]{\begin{equation*}#1\end{equation*}}
\newcommand{\eql}[1]{\begin{equation}#1\end{equation}}
\newcommand{\pd}[2]{ \frac{ \partial #1}{\partial #2} }
\newcommand{\NN}{\mathbb{N}}
\newcommand{\RR}{\mathbb{R}}
\newcommand{\CC}{\mathbb{C}}
\usepackage{siunitx}
%\renewcommand{\cite}{\citep}%
\newcommand{\ms}{\si{\milli\second}}%
\newcommand{\m}{\si{\meter}}%
\newcommand{\s}{\si{\second}}%
\usepackage[natbib=true,
			style=ieee, %numeric-comp,
      		sortcites=true,
			abbreviate=true,
			maxcitenames=2,
			maxnames = 5,
%			firstinits=true,
%			uniquename=init,
%			sorting=none,
			doi=false,
			url=true,
			isbn=false,
			eprint=false,
			texencoding=utf8,
			bibencoding=latin1,
			%autocite=superscript,
			backend=bibtex,
			%articletitle=false,
			]{biblatex}%
\AtEveryBibitem{
  \clearfield{month}
  \clearfield{day}
  \clearfield{url}
  \clearfield{note}
  \clearfield{comment}
%  \clearfield{edition}
%  \clearfield{publisher}
}
\addbibresource{hulk.bib}%

\newcommand{\coef}{\mathbf{a}} % image's hidden param
\newcommand{\image}{\mathbf{I}} % the image
\newcommand{\dico}{\Phi} % the dictionary
%\showthe\columnwidth
%
% paper title
% can use linebreaks \\ within to get better formatting as desired
%\usepackage{titling}
%\setlength{\droptitle}{-6em}
%\posttitle{\par\end{center}\vspace{-2.3em}}

\title{\Title}


% author names and affiliations
% use a multiple column layout for up to three different
% affiliations
\author{%
%\makebox[.45\linewidth]{\FirstNameA\ \AuthorA}
%\and \makebox[.45\linewidth]{\FirstNameB\ \AuthorB}% \\ \Institute\ \Organism
%\makebox[.9\linewidth]{\FirstNameA\ \AuthorA and \FirstNameB\ \AuthorB} %\\ \Institute\
\FirstNameA\ \AuthorA \and \FirstNameB\ \AuthorB \and \FirstNameC\ \AuthorC
}
\date{\Institute\ \\ 
\Organism\
}

% The \author macro works with any number of authors. There are two
% commands used to separate the names and addresses of multiple
% authors: \And and \AND.
%
% Using \And between authors leaves it to LaTeX to determine where to
% break the lines. Using \AND forces a line break at that point. So,
% if LaTeX puts 3 of 4 authors names on the first line, and the last
% on the second line, try using \AND instead of \And before the third
% author name.

\author{
David S.~Hippocampus\thanks{Use footnote for providing further
information about author (webpage, alternative
address)---\emph{not} for acknowledging funding agencies.} \\
Department of Computer Science\\
Cranberry-Lemon University\\
Pittsburgh, PA 15213 \\
\texttt{hippo@cs.cranberry-lemon.edu} \\
%% examples of more authors
%% \And
%% Coauthor \\
%% Affiliation \\
%% Address \\
%% \texttt{email} \\
%% \AND
%% Coauthor \\
%% Affiliation \\
%% Address \\
%% \texttt{email} \\
%% \And
%% Coauthor \\
%% Affiliation \\
%% Address \\
%% \texttt{email} \\
%% \And
%% Coauthor \\
%% Affiliation \\
%% Address \\
%% \texttt{email} \\
}

%\setlength{\pdfpagewidth}{\paperwidth}
%\setlength{\pdfpageheight}{\paperheight}



\usepackage{times}

\usepackage{graphicx}
\DeclareGraphicsExtensions{.pdf,.png,.jpg}

%\usepackage{fancyhdr}


%\usepackage[top=2cm, bottom=2cm, left=2cm, right=2cm]{geometry}
%\usepackage[margin=0.5in]{geometry}
%\oddsidemargin 0.0in		% margin, in addition to 1" standard
%\textwidth 6.5in		% 8.5" - 2*(1+\oddsidemargin)
%
%%\topmargin -0.25in		% in addition to 1.5" standard margin
%\topmargin -.75in		% in addition to 1.5" standard margin
%\textheight 8.75in 		% 11 - ( 1.5 + \topmargin + <bottom-margin> )
%
%\columnsep 0.1in
%
%\parindent 0pt
%\parskip 12pt
%
%\flushbottom \sloppy
%\topskip    = 0.0truein
%\topmargin  = 0.0truein
%\headheight = 0.0truein
%\headsep    = 0.0truein
%\hoffset    = -0.48truein
%\voffset    = -0.9truein
%\parskip    = \baselineskip
%\parindent  = 0pt
%\textwidth  = 6.4truein
%\textheight = 10.1truein
%\flushbottom
%Submission Format
%
%Before you log onto the submission website, you should have the following items prepared. Submissions that do not meet these guidelines may be rejected. Abstract selection is a competitive process, so please read carefully:
%
%    * Title - 100 characters or fewer (including spaces), capitalized in sentence case
%    * Author list - including email addresses of all authors
%    * Summary - 300 words or fewer, text only. This summary will appear in the printed conference program if your submission is accepted.
%    * PDF submission - This should be one page (A4 or US Letter) in PDF format. The title and author names should appear at the head of the page (contact info may be omitted), followed by the 300-word Summary. The rest of the document should expand upon the central question(s), approach, results, and/or conclusions of the study. Figures are optional. You may include equations as appropriate. Font size (including any figure legends) must be at least 12 point. Margins should be at least 0.5". Because space is limited, you do not need to touch upon all the major points of the Summary; rather you should aim to add whatever detail is need to help reviewers evaluate the technical correctness and significance of your study.
%    * Two Program Committee members (see below) who could handle your submission. We cannot honor all specific requests, but your list will help us direct your submission to Program Committee members with appropriate expertise. 
%
%Evaluation Criteria
%
%PDF submissions will be evaluated on the basis of the following criteria (equally weighted):
%
%    * Significance, originality, and potential impact of your claims to the COSYNE audience. Please indicate why your question is important and how your claims advance the field. Reviewers will be asked to rate your claims in this regard, without regard to whether your methods/results technically support those claims (see below). If your aim is to develop a new approach or technique, indicate why this is useful. Implicitly included in this category is the quality of fit of your abstract for the COSYNE conference audience -- potentially inappropriate abstracts include pure machine learning studies, or studies of single cells with no clear implications for neural systems.
%    * Technical correctness. Reviewers will be asked to rate the quality of your methods and/or data in supporting your claims (see above). Thus, you should clearly outline the methods used, results obtained, and critical controls. Include sufficient detail in your submission to allow reviewers to evaluate whether your approach is appropriate for supporting your claims, and whether your results are technically sound. However, your submission should not be as detailed as the Methods section of a full-length paper (e.g., do not list concentrations of drugs, etc.). 
%
%You should not feel obliged to fill the entire page, and we expect that many successful submissions will not include figures.
%
%Approximately 20 submissions will be chosen for short talks and ~300 will be chosen for poster presentations. We expect ~20% of submissions will be rejected due to limitations on poster space. 

\makeatletter

\pagestyle{empty}		% No page numbers
\begin{document}
%
% make the title area
\maketitle
%%%----------------------------------------------------------------- 
%{\LARGE\bf  
%Title of Our Abstract
%}
%
%{\Large\bf
%First Author$^1$, Second Author$^2$, and Third Author$^1$ 
%}
%
%{\Large
%$^1$Renowned University, \hspace{1em} $^2$Another University
%}

%%%----------------------------------------------------------------- 
\vspace*{-.6cm}
%\begin{abstract}
{\bf Abstract.}
\Abstract
\vspace*{-.4cm}
%\end{abstract}
\thispagestyle{empty}

%The full text of the abstract, including any equations, tables, and
%figures, should appear here.  Left  margin is 1 inch (2.54 cm), and
%text width is 6.75 inches (16.5 cm), which leaves a 1 inch right
%margin on U.S.-letter sized paper.  Text is full-justified (i.e., both
%right and left side are aligned).
%
%Paragraphs should be separated by a blank line, and the first line 
%should not be indented.  
%
%\Figure
%\ParagIntro

\section{Introduction}\label{introduction}

The neural architecture is a complex dynamic system that operates at
different time scales. In particular, one of its properties is to
succeed in the feat of being able to both represent information quickly
but also to be able to adapt in the long term autonomously
(self-organized) to optimize this coding. In the case of the mammalian
primary visual cortex (V1) for instance, one can observe the rapid
coding of retinal image, as a process of transforming the visual
information into a rough sketch that represents the outlines of objects
in the image. This rapid operation, of the order of 50 milli-seconds in
humans, is key to the results of~\textcite{hubel1968receptive}, who showed that some
cells of the mammalian primary visual cortex have relatively localized
receptive fields which are predominantly selective at different
orientations. A major step in understanding this observation has been to
show that the emergence of these filters can be explained as the
coupling of a simple Hebbian learning with an optimal coding of the
image. Indeed, the work of~\textcite{olshausen1996emergence} has shown that by imposing a
sparse encoding of the image, we can obtain the emergence of such cells
in a neural-type model. This type of unsupervised learning algorithm~\citep{Olshausen97} has
been related to many other types of optimal representation algorithms
used in both signal processing and artificial intelligence. In
particular, this algorithm allows the extraction of the independent
components of the signal, for example the orientations in the image.
This representation makes it possible to observe the emergence of
representations which are invariant to geometric transforms such as
rotations and translations in V1. A recent rapprochement between these
algorithms and machine learning algorithms makes it possible to place
them in a new perspective. 
%https://blog.keras.io/building-autoencoders-in-keras.html Otherwise, one reason why they have attracted so much research and attention is because they have long been thought to be a potential avenue for solving the problem of unsupervised learning, i.e. the learning of useful representations without the need for labels. Then again, autoencoders are not a true unsupervised learning technique (which would imply a different learning process altogether), they are a self-supervised technique, a specific instance of supervised learning where the targets are generated from the input data. 
Indeed, these unsupervised algorithms are equivalent to a gradient descent optimization over an informational-type
coding cost. This cost makes it possible to quantitatively evaluate the
exploration of new components in the signal for learning with respect to
the exploitation of these components in the coding. As such, this remark
shows us that unsupervised learning consists of two antagonistic
mechanisms, a long time scale that corresponds to the learning and
exploration of new components and a faster scale that corresponds to
coding.

%\newcommand{\ParagIntro}{%
%It is observed that simple cell neurones in mammalian primary visual cortex are selective to orientation, spatial localisation, and frequencies~\citep{hubel1968receptive}. It is demonstrated that developing a coding strategy that maximises sparseness is sufficient to form receptive fields that account for all three of the above properties~\citep{olshausen1996emergence}. Visual items composing natural images are often sparse, such that the brain may use this sparseness to reconstruct images with only a few set of these items~\citep{Perrinet10shl,Perrinet15bicv}. This is supporting the idea that an unsupervised learning algorithm based on sparse coding could be use to describe efficiently image processing in the primary visual cortex. Llearning is accomplished in {\sc SparseNet}~\citep{Olshausen97} on patches taken from natural images as a sequence of coding and learning steps. First, knowing a dictionary of receptive fields $\dico_i$, the sparse coding is achieved using a gradient descent over a convex cost derived from a sparse prior probability distribution function of the coefficients $a_i$, where we use the generic $\ell_0$ norm sparseness, by simply counting the number of non-zero coefficients:
%\begin{equation}%
%\mathcal{C}_0( \coef | \image , \dico) = \frac{1}{2\sigma_n^2} \| \image - \dico \coef \|^2 + \lambda \| \coef \|_0 \nonumber%
%\end{equation}%
%During the early learning phase, some cells may learn ``faster'' than others. There is the need for a homeostasis mechanism that will ensure convergence of learning. The goal of this work is to study the specific role of homeostasis in learning sparse representations and to propose a homeostasis mechanism which optimises the learning of an efficient neural representation.%
%}%
%\newcommand{\ParagLongIntro}{%
%It is observed that simple cell neurones in mammalian primary visual cortex are selective to orientation, spatial localisation, and frequencies~\citep{hubel1968receptive}. It is demonstrated that developing a coding strategy that maximises sparseness is sufficient to form receptive fields that account for all three of the above properties~\citep{olshausen1996emergence}. Visual items composing natural images are often sparse, such that the brain may use this sparseness to reconstruct images with only a few set of these items~\citep{Perrinet15bicv}. This is supporting the idea that an unsupervised learning algorithm based on sparse coding could be use to describe efficiently image processing in the primary visual cortex.
%
%Most of existing models of unsupervised learning aim at optimising a cost defined on prior assumptions on representation's sparseness. For instance, learning is accomplished in {\sc SparseNet}~\citep{Olshausen97} on patches taken from natural images as a sequence of coding and learning steps. First, knowing a dictionary of receptive fields $\dico_i$, the sparse coding is achieved using a gradient descent over a convex cost derived from a sparse prior probability distribution function of the coefficients $a_i$. Then, knowing this sparse solution, learning is defined as slowly changing the dictionary using Hebbian learning. In general, the parameterisation of the prior has major impacts on results of the sparse coding and thus on the emergence of edge-like receptive fields and requires proper tuning. In fact, the definition of the prior corresponds to an objective sparseness and does not always fit to the observed probability distribution function of the coefficients. In particular, this could be a problem \emph{during} learning if we use the cost to measure representation efficiency for this learning step. An alternative is to use a more generic $\ell_0$ norm sparseness, by simply counting the number of non-zero coefficients:
%\begin{equation}%
%\mathcal{C}_0( \coef | \image , \dico) = \frac{1}{2\sigma_n^2} \| \image - \dico \coef \|^2 + \lambda \| \coef \|_0 \nonumber%
%\end{equation}%
%It was found that by using an algorithm like Matching Pursuit, the learning algorithm could provide results similar to {\sc SparseNet}, but without the need of parametric assumptions on the prior~\citep{Perrinet10shl}. However, we observed that this class of algorithms could lead to solutions corresponding to a local minimum of the objective function: Some solutions seem as efficient as others for representing the signal but do not represent edge-like features homogeneously. In particular, during the early learning phase, some cells may learn ``faster'' than others. There is the need for a homeostasis mechanism that will ensure convergence of learning. The goal of this work is to study the specific role of homeostasis in learning sparse representations and to propose a homeostasis mechanism which optimises the learning of an efficient neural representation.%
%
%To achieve this, we first formulate analytically the problem of representation efficiency in a population of sensory neurones. For the particular $\ell_0$ norm sparseness, we show that sparseness is optimal, in term of Shannon entropy, when average activity within the neural population is uniformly balanced (i.e. each neurone is selected with the same probability when encoding a large set of data). To achieve this uniformity, we define an homeostatic gain control mechanism based on histogram equalisation, that is in transforming coefficients in terms of z-scores $z_i(a_i) = P( \cdot > a_i)$. The cumulative distribution $z_i$ for each coefficient of the sparse vector is calculated using Hebbian learning to smooth its evolution during learning. At the coding level, this z-score function is incorporated in the matching step of the matching pursuit algorithm, to modulate the choice of the most  as that with the maximal z-score: $i^\ast = \mathrm{Argmax}_i z_i(a_i)$. The rest of the algorithm is left unchanged.
%
%We compared qualitatively the set $\dico$ of receptive filters generated by the proposed algorithm when the homeostasis is first turned-off and then enabled  (see Fig.~\ref{fig:map}). A more quantitative study of the coding is shown by comparing selection distribution of sparse coefficients when the homeostasis mechanism is turned on (see Fig.~\ref{fig:quant}). We demonstrate that forcing the learning activity to be uniformly spread among all receptive fields results in a faster convergence of the representation error, and in an increase of the Shanon entropy. Finally, an interesting perspective is to apply the homeostatic regulation algorithm in a classical fully connected deep-learning neural network and applied on the MNIST recognition task. By using the sparse coefficients as the input layer of the network, we can compare the performance obtained with and without the homeostatic mechanism. Preliminary results show that the improvement in efficiency is more acute when using sparse representations (5 out of 324 coefficients).
%}%



A component often ignored in this type of learning is the set of
mechanisms of homeostasis. Indeed, these are implemented in the original
algorithms (Olshausen, Rehn, \ldots{}) as a heuristic that simply
prevents the algorithm from diverging. It consists in some recent
algorithms to constitute only of an equalization of the energy of each
component of the dictionary (Mairal). However, the neural mechanisms of
homeostasis are at work in many components of the neural code and are
essential to the overall functioning of the neural code. For example,
the networks of GABA-type inhibitory neurons make it possible to
regulate the overall activity of neural populations. This mechanism then
makes it possible to balance the contribution of the excitatory
populations with respect to that in inhibitory populations. By this
mechanism, this type of so-called balanced networks can explain many
features inherent in the primary visual cortex (Hansel). These
mechanisms, which often take the form of normalization rules (Schwartz)
in the network, are often used as normative theory to explain the
mechanisms present in the primary visual cortex (see Heeger). However,
this work is often intended to show that the cells that emerge from
these algorithms have the same characteristics as that observed in
neurophysiology (Rehn, Loxley). Other algorithms use nonlinearities that
implicitly implement homeostatic rules in neuromimic algorithms
(Gerstner and Brito). These non-linearities are mainly used in the
output of successive layers of deep learning networks that are nowadays
widely used for image classification or artificial intelligence. However
most of these non-linear normalization rules are based on heuristics
mimicking neural mechanisms but are not justified as part of the global
problem of unsupervised learning.

It is important to note at this point that learning algorithms used in
artificial intelligence can shed new light on the functioning of these
neural processes and in particular on unsupervised learning. In the
field of machine learning, unsupervised learning corresponds to learning
representation dictionaries when the categorization data is unknown.
This training is therefore carried out autonomously and is particularly
used for image and signal compression, object detection, source
separation and noise reduction in raw signals. As such, this class of
algorithms is extremely useful in the first layers of artificial
intelligence algorithms such as deep algorithms. There are many variants
of such algorithms that take the form of either an optimization of
information transfer (Bell), or in the form of learning rules in deep
learning or recursive algorithms in statistics and probabilities with,
for example, projection continuation. An important class of these
algorithms considers that all the solutions that will be considered are
those that correspond to an optimal coding. These solutions take the
form of parsimonious coding, i. e. for which a small number of
components will be selected according to the size of the dictionary.
This principle is general enough to be applied to many signal classes
and allow a mathematical analysis of this problem (Donoho). In
particular, we have shown above that a rapid algorithm of parsimonious
coding can be implemented in a normal neural architecture (Perrinet,
2010). It has also been shown that such coding can improve
classification algorithms, in particular by limiting the number of
layers required in a deep learning algorithm for classifying images that
contain or do not contain animals (Perrinet and Bednar, 2015). However,
recent studies seem to question this principle of parsimonious coding
(Eichhorn 2009, Zoran and Weiss 2012) and suggest that simpler analysis
applied in a complex metric is sufficient to explain the emergence of
selective filters with orientation similar to those observed in the
primary visual cortex.

In order to offer a broader perspective on this problem, we will try to
express it in the generic form of a probabilistic problem. This approach
is already widely used in Barlow's early work under the term of
redundancy reduction principle (see also Atick). It led to translating
this learning problem into a problem of efficient coding, for example by
implementing inhibition rules in the retinal receptive field of the
saber-toothed tiger (??? Srinivasan, 1981). Other studies show that
these rules are tantamount to forcing the system to be close to a
criticality regime and optimizing the balance between coding and
learning optimization (Beggo, 2008 in Sandin) More generally, we will
place ourselves in the framework of the principle of free energy
minimization formulated by Karl Friston. This principle makes it
possible to explicitly address the problem of coding and unsupervised
learning, also during learning. In this theory, learning is no longer a
goal in itself but contributes to the the minimization of free energy at
different time scales, from coding to learning but also to the
intermediate time of homeostasis. According to this principle, the
overall goal of neural system is to be able to best predict any sensory
input. This principle results in changes in the structure of the
population (synaptic connections) but also in adaptation rule before the
convergence of the learning. The goal of these processes is thus simply
to not be surprised by the sensory input. As such, the aim of learning
is to obtain a coding that is the least variable a priori, that is to
say, before having received the sensory information. On the one hand,
this theory extends that of Olshausen and shows that homeostasis also
has a predictive part. Thus, this will also allow us to formulate a
normative theory of the neural code at the time scale of homeostasis.
This allows us also to associate homeostasis with adaptation mechanisms
(Rao Balard). Thus, the set of processes at different time scales are
thus considered as working synergestically and provide for a novel
normative theory of coding in early sensory areas such as V1.

This paper is organized according to the following plan. First we will
show the importance of homeostasis in unsupervised learning algorithms.
We will derive an optimal rule for homeostatic adaptation based on
histogram equalization. We will then show quantitative results of this
optimal algorithm by applying it to different pairs of coding and
learning algorithms. By using different learning databases, we will be
able to give a quantitative analysis that will make it possible to
compare these different solutions. To simplify the optimal rule of
homeostasis, we will then deliver a neuro-mimetic homeostasis algorithm
derived from the optimal rule by using a simple heuristic. We will then
compare the results of this new algorithm with the optimal algorithm as
well as with other existing unsupervised learning algorithms (Olshausen,
Sandin). Moreover, by its nature, this algorithm can easily be extended
to convolutional networks such as those used in deep learning neural
networks. This extension is possible by extending the filter dictionary
by the hypothesis of invariances to the translation of representations.
Our results on different databases show the stable and rapid emergence
of characteristic filters on these different bases. This result shows a
probable prospect of extending this representation and for which we hope
to obtain classification results superior to the algorithms existing in
the state-of-the-art. Finally we will conclude by showing the importance
of homeostasis in unsupervised learning algorithms. By developing this
fast learning algorithm we hope for its rapid application in artificial
intelligence algorithms. This type of architecture is economical,
efficient and fast. It makes it possible to be transferred to most deep
learning algorithms, a major flaw of which is to be very greedy in
computing resources. In particular, we are considering perspectives for
coding within a dynamic flow of sensory data and we hope to apply this
type of algorithm on embedded systems such as aerial robots. Along with
this, we hope that this new type of rapid unsupervised learning
algorithm can provide a normative theory for the coding of information
in low-level sensory processing, whether it is visual or auditory, for
example.

\section{Algorithm}\label{algorithm}




\subsection{Objective}\label{objective}

% Link with VAE : https://wiseodd.github.io/techblog/2016/12/10/variational-autoencoder/

Knowing a dictionary of receptive fields $\dico_i$~\citep{hubel1968receptive}, we achieve sparse coding using a gradient descent~\citep{olshausen1996emergence,Olshausen97} over a  cost derived from a sparse prior probability distribution function of the coefficients $a_i$, where we use the generic $\ell_0$ norm sparseness~\citep{Perrinet10shl,Perrinet15bicv}, by simply counting the number of non-zero coefficients:
\begin{equation}%
\mathcal{C}_0( \coef | \image , \dico) = \frac{1}{2\sigma_n^2} \| \image - \dico \coef \|^2 + \lambda \| \coef \|_0 \nonumber%
\end{equation}%
During the early learning phase, some cells may learn ``faster'' than others. There is the need for a homeostasis mechanism that will ensure convergence of learning. In particular, we will set the a priori probability of selecting coefficients {\color{BrickRed} $\forall (i,j), P(\coef_i)=P(\coef_j)$} to ensure the optimality of the choice of the  pseudo $\ell_0$-norm and compare it to the representation in the primary visual cortex.%

\subsection{Algorithm}\label{algorithm}


The proposed algorithm is:
\begin{enumerate}%
{\color{MidnightBlue} 
\item Initialize the point non-linear gain functions $z_i$ to similar cumulative distribution functions and the components $\dico_i$ to random points on the unit $L$-dimensional sphere,%
\item repeat until learning converged:%
\begin{enumerate}%
{\color{OliveGreen}
	\item draw a signal $\image$ from the database, its energy is $E = \| \image \|^2$,%
	\item set sparse vector $\coef$ to zero, initialize $\bar{a}_i=<\image, \dico_i >$ for all $i$,% using~\seeEq{coco},%
	\item while the residual energy $E$ is above a given threshold do:
	\begin{enumerate}
		{\color{BrickRed}
			\item select the best match: $i^\ast = \mbox{ArgMax}_{i} [z_i( \bar{a}_i )]$,% with~\seeEq{mp1},
		}
		\item set the sparse coefficient: $a_{i^\ast} = \bar{a}_{i^\ast}$,
		\item update residual coefficients: $\forall i, \bar{a}_i \leftarrow \bar{a}_i - a_{i^\ast} <\dico_{i^\ast} , \dico_i > $,% for all $i$ using~\seeEq{mp3},
		\item update energy: $E \leftarrow E - a_{i^\ast}^2 $.
	\end{enumerate}
}
\item when we have the sparse representation vector $\coef$, apply $\forall i$:
\begin{enumerate}
\item modify dictionary: $\dico_{i} \leftarrow \dico_{i} + \eta a_{i} (\image - \dico\coef)$,% using~\seeEq{learn}, 
\item normalize dictionary: $\dico_{i} \leftarrow \dico_{i} / \| \dico_{i}\|$,% using~\seeEq{learn}, 
{\color{BrickRed}
\item update homeostasis functions: $z_i( \cdot ) \leftarrow (1- \eta_h ) z_i( \cdot ) + \eta_h \delta( a_i \leq \cdot)$.% using~\seeEq{learn_homeo}
}
\end{enumerate}
\end{enumerate}
}
\end{enumerate}


% LUP\ IS\ HERE\ 

Le principe de l'algorithme consiste à rajouter des variables latentes
qui vont permettre de prédire les déviations dans le codage ou
l'apprentissage pendant l'apprentissage.

\section{Methods}\label{methods}


\section{Results}\label{results}


\section{Discussion and conclusion}\label{discussion-et-conclusion}


% Finally, we show that such an algorithm can be extended to convolutional architectures and we show the results obtained on different natural image databases. 
%
%One core advantage of sparse representations is the efficient coding of complex signals using compact codes. For instance, it allows for the representation of any sample as a combination of few elements drawn from a large dictionary of basis functions. In the context of the efficient processing of natural images, we propose here that sparse coding can be optimized by designing a proper homeostatic rule regulating the competition between the elements of the dictionary. Indeed, a common design for unsupervised learning rules relies on a gradient descent over a cost measuring representation quality with respect to sparseness. The sparseness constraint introduces a competition which can be optimized by ensuring that each item in the dictionary is selected as often as others. We implemented this rule by introducing a gain normalisation similar to what is observed in biological neural networks. We validated this theoretical insight by challenging the matching pursuit sparse coding algorithm with the same learning rule but with or without homeostasis. Simulations show that for a given homeostasis rule, gradient descent performed similarly the learning of a dataset of image patches. While the coding accuracy did not vary much, including homeostasis changed qualitatively the learned features. In particular, homeostasis results in a more homogeneous set of orientation selective filters, which is closer to what is found in the visual cortex of mammals. To further validate these results, we will apply this algorithm to the optimisation of a visual system to be embedded in an aerial robot. In summary, this biologically-inspired learning rule demonstrates that principles observed in neural computations can help improve real-life machine learning algorithms. 
 %The different sparse coding algorithms were chosen for their efficiency and generality. They include least-angle regression, orthogonal matching pursuit and basis pursuit. 

%2. importance of homeo - vanishing term in deep learning -> use deep learning to validate output
%3. application to asynchronous / focal  log-polar (retinal) input / continuous learning / credit assignement (no access to true residual)

%To further validate these results, we applied this algorithm to the optimization of a visual system embedded in an aerial robot. Indeed, such robots have an increasing demand for visual applications, such as computation of optic flow for flight stabilization or for common visual categorization tasks such as recognition of pilot's gestures. However, embedding such a system is often difficult as it necessitates to be rapid,  energy efficient and lightweight. We resolved this problem by introducing our novel algorithm as an output of our visual sensors (the ``retina''). Results were collected by comparing the image reconstruction error for different information bandwidths in the output bus of this retina.Quantitative results show that the quality of the reconstruction was improved in our algorithm on a  database of natural, flying videos of the aerial robot.In particular, this algorithm provides with a real-life example showing how to improve our understanding of the emergence of edge-selective simple cells, drawing the bridge between structure (representation) and function (efficient coding).% 
%

%\FigureMap
%\columnbreak
%tata
%\Figure
%\FigureQuant

%%%%%----------------------------------------------------------------- 
%%\section*{Acknowledgments}
%%%{\bf Acknowledgments}\\
%%\Acknowledgments

%%%----------------------------------------------------------------- 
%{\bf References} \\
\printbibliography
%Column 1
%\columnbreak
%Column 2

\end{document}